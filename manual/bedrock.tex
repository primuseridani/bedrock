\documentclass[a5paper, twoside]{report}

\usepackage[english]{babel}
\usepackage{fancyhdr}
\usepackage[T1]{fontenc}
\usepackage[margin=2cm]{geometry}
\usepackage[hidelinks, pdfusetitle]{hyperref}
\usepackage{lastpage}
\usepackage{mdframed}
\usepackage{multicol}
\usepackage{parskip}
\usepackage{tgtermes}
\usepackage[center]{titlesec}
\usepackage{varwidth}

\author{Achernar}
\title{The Bedrock Manual}
\date{2025-04-07}

\renewcommand{\chaptermark}[1]{\markboth{#1}{}}

\pagestyle{fancy}
\fancypagestyle{plain}{\pagestyle{fancy}}
\fancyhf{}
\fancyhead[lo,re]{\small\bfseries\thechapter.\ \MakeUppercase{\leftmark}}
\fancyfoot[lo,re]{\small\bfseries \thepage\ / \pageref*{LastPage}}

\titleformat{\chapter}{\flushright\huge\bfseries}{}{0pt}{\thechapter.\ }[]
\titleformat{\section}{\large\bfseries}{}{0pt}{\thesection.\ }[]
\titleformat{\subsection}{\centering\small\bfseries}{}{0pt}{\thesubsection.\ }[]

\titlespacing*{\chapter}{0pt}{0pt}{0pt}
\titlespacing*{\section}{0pt}{0.5cm}{0.5cm}
\titlespacing*{\subsection}{0pt}{0.25cm}{0.25cm}

\setlength{\columnsep}{1cm}

\newcommand*{\rulewidth}{0pt}
\renewcommand*{\headrulewidth}{\rulewidth}
\renewcommand*{\footrulewidth}{\rulewidth}

\begin{document}
	\thispagestyle{empty}
	\pagenumbering{gobble}

	\begin{center}
		\vspace*{\fill}
		{\large\bfseries The}

		{\huge\bfseries Bedrock Manual}

		{\large\bfseries 0.5}

		\vspace*{\fill}
		{\small Copyright \textcopyright\ 2025 Gabriel Bjørnager Jensen}
	\end{center}

	\clearpage
	\pagenumbering{arabic}

	\tableofcontents
	\addcontentsline{toc}{chapter}{\numberline{\thechapter}\contentsname}

	\clearpage
	\chapter{Introduction}
		\textit{Bedrock} is an artillery video game developed at Achernar.

		This manual describes the basics of the game, including how to set it up, how gameplay works, and more.

		\section{Technical specifications}
			\begin{mdframed}
				\small\itshape\bfseries
				Note: Bedrock is still early in its development, and not all platforms may yet have been thoroughly tested.
			\end{mdframed}

			\subsection{Machine architecture}
				Bedrock requires at least \texttt{32} bits per memory address.
				So far, Bedrock has been tested to work on AMD64 (x86-64) machines, and support is therefore also expected on IA-32 (specifically i686-compatible variants).
				Other architectures have not been tested yet.

			\subsection{Operating system}
				Bedrock is written to support UNIX-compatible systems as well as Microsoft Windows.
				So far, however, only Linux has been tested (specifically Arch Linux).

			\subsection{Graphics interface}
				Bedrock can make use of any of the following windowing protocols, depending on the target operating system:

				\begin{multicols}{2}
					\begin{itemize}
						\item{Cocoa}
						\item{Orbital}
						\item{X}
						\item{Wayland}
						\item{Win32}
					\end{itemize}
				\end{multicols}

				Additionally, rendering is done using any of the following interfaces:

				\begin{multicols}{2}
					\begin{itemize}
						\item{Direct3D}
						\item{Metal}
						\item{OpenGL}
						\item{Vulkan}
					\end{itemize}
				\end{multicols}

		\section{Contact}
			For support regarding Bedrock, please write an e-mail to Achernar at:

			\begin{mdframed}
				\ttfamily
				gabriel@achernar.io
			\end{mdframed}

	\clearpage
	\chapter{Installation}
		This chapter will go over the different ways of installing Bedrock.

		\section{Arch Linux}
			Use the official PKGBUILD file at:

			\begin{mdframed}
				\url{https://mandelbrot.dk/pkgbuild/bedrock.git}
			\end{mdframed}

		\section{From sources}
			This section will guide you on how to install Bedrock from its sources.
			Instructions must be done in a terminal emulator.

			\subsection{Prerequisites}
				The following, other programmes must be installed on the host system when building Bedrock from sources:

				\begin{multicols}{2}
					\begin{itemize}
						\item{Git}
						\item{The rustc compiler}
						\item{The Cargo build system}
					\end{itemize}
				\end{multicols}

				See \url{https://git-scm.com/downloads/} on how to install Git.
				See \url{https://www.rust-lang.org/} for information on how to install the Rust programming language.

			\subsection{Getting the sources}
				The repositories containing the sources for Bedrock may be any of:

				\begin{mdframed}
					\url{https://mandelbrot.dk/achernar/bedrock.git}

					\url{https://gitlab.com/primuseridani/bedrock.git}

					\url{https://github.com/primuseridani/bedrock.git}
				\end{mdframed}

				Instruct Git to download the desired repository:

				\begin{mdframed}
					\ttfamily
					git clone -{}-depth 0 \textit{repository}
				\end{mdframed}

				... wherein \textit{repository} is the address of the Bedrock repository.
				Git will then unpack the repository into a \texttt{bedrock} directory relative to the current working directory.

			\subsection{Building}
				Once downloaded, the sources must be compiled into the final executable.

				Change the working directory into the newly-created \texttt{bedrock} directory:

				\begin{mdframed}
					\ttfamily
					cd bedrock
				\end{mdframed}

				And then invoke Cargo via:

				\begin{mdframed}
					\ttfamily
					cargo build -{}-release
				\end{mdframed}

				When compilation has finished -- which can take quite some time -- the final binary should be located inside a new \texttt{target} directory.

			\subsection{Installation, UNIX\textsuperscript{\textregistered}}
				Finally, the installation script \texttt{install.sh} can be used to install the executable, along with some other files:

				\begin{mdframed}
					\ttfamily
					./install.sh \textit{base\_directory}
				\end{mdframed}

				... wherein \texttt{\itshape base\_directory} denotes a custom root directory.
				The script will then populate the following subdirectories of the specified root directory:

				\begin{itemize}
					\ttfamily

					\item{usr/bin}
					\item{usr/share/applications}
					\item{usr/share/pixmaps}
				\end{itemize}

				\begin{mdframed}
					\small\itshape\bfseries
					Hint: Remember to take note of the script's output, as this may be needed when uninstalling Bedrock.
				\end{mdframed}

			\subsection{Installation, Windows\textsuperscript{\textregistered}}
				Currently, Bedrock does not provide an out-of-the-box, automated means of installation on Microsoft Windows systems.

	\clearpage
	\chapter{Acknowledgements}
		\section{Bedrock}
			Bedrock is copyright \copyright\ 2025 Gabriel Bjørnager Jensen.
			All rights reserved.

		\section{The Bedrock Manual}
			This manual is copyright \textcopyright\ 2025 Gabriel Bjørnager Jensen.
			It is licensed under a Creative Commons Attribution-ShareAlike 4.0 International licence.
			See \url{https://creativecommons.org/licenses/by-sa/4.0/} for more information.

		\section{Trade marks}
			Cocoa, \nobreak{MacOS}, and Metal are registered trade marks of Apple Inc. in the E.U.;
			Linux is a registered trade mark of Linus Benedict Torvalds in the E.U.;
			OpenGL is a registered trade mark of Hewlett Packard Enterprise Development LP in the E.U.;
			UNIX and X are registered trade marks of the Open Group Limited in the E.U.;
			Vulkan is a registered trade mark of the Khronos Group Inc. in the E.U.;
			Windows is a registered trade mark of Microsoft Corporation in the E.U.
\end{document}
